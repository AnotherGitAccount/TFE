\chapter{Conclusion}

The goal of this work was to design a CPU, the beta machine, for the laboratories of the Computer 
Structures [INFO0012] 
course at the University of Liege. As described in this document, this has been done completely. In 
addition to that, access to some IOs as well as a graphical accelerator called GPU in the work has been
implemented. They allow to add a lot of interraction for the students. In addition to that, it has 
been made so that the values in different memories can be written and read from the ARM processor. 
The ARM processor was chosen as the access to the system developed in this work in order to simplify 
the access. Indeed, it is enough to know how to navigate in a linux terminal and to launch a program 
to be able to use the system and the beta machine, which is very simple. The whole work was done 
keeping in mind that it should be simple to use this system. Finally, several demonstrations were 
made to show that everything worked as it should.

This work has therefore fully met its objectives. However, several things could be improved in future 
work. First, it would be possible to greatly increase the performance of the processor by pipelining 
its different paths and modules. Just by removing the execution sequence, the troughput would be 
multiplied by 7. Concerning pipelining, if it is done at all levels (especially for complex operations 
of the alu: multiplication and division), it would increase the clock frequency of the CPU. With 
this done correctly, it is possible to go up to 100 or 200MHz, against the current 50MHz. Then, it 
is possible to add more IOs. At least managing all the ones on the card and adding audio management 
on the HDMI would be a nice addition. Finally, modifications to the software on the ARM side and 
memory access on the FPGA side could allow debugging with breakpoints in the code etc., which could 
be really interesting for the labs. 

Although the system is already functional and easy to use, there are still many possibilities for 
modifications. The only limit is the capacity of the FPGA, which could be changed anyway if needed.