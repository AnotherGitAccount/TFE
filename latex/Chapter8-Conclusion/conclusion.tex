\chapter{Conclusion}

The goal of this work was to design a harvard 32bits CPU, the beta machine, for the laboratories of the Computer 
Structures [INFO0012] 
course at the University of Liège. As described in this document, this has been done completely. 
Indeed, the beta machine is implemented with arithmetic, shift, conditional branch and jump operations. As 
it is a Harvard architecture, it has two memories: one for data (65kB) and one for instructions (131kB). In addition to 
all this, a register file provides 32 registers of 32 bits to the machine. This is more than enough for many applications.

Access to some IOs as well as a graphical accelerator called GPU in the work has been
implemented. They allow to add a lot of interraction for the students. 
The IO access unit is easily accessible through the CPU Load and Store operations and allows the 
reading and/or writing of three IOs: two push buttons, four switches and eight LEDs. Concerning the 
GPU, it allows rendering on a 16:9 60Hz screen with a resolution of 848x480 pixels (of which 
576x432 can be written). The writing is done using masks that the user can program. The masks 
allow to write 8x8 pixels per operation with two different colors (it can also be chosen to clear 
a pixel or to keep its current value). This allows for a more efficient GPU, in fact in the best 
case it accelerates the writing by 64, which is a great saving in clock cycles.

In addition, a facility to write and 
read the values in different memories from the ARM processor has been provided. 
The ARM processor was chosen as the access to the system developed in this work in order to simplify 
the access.  Thanks to this, it is enough to know how to navigate in a Linux terminal and to launch a program 
to be able to use the system and the beta machine, which is very simple. The whole work was done with simplicity 
in mind.  Finally, several demonstrations were 
made to show that everything worked as it should.

This work has therefore fully met its objectives. However, several things could be improved in future 
work. First, it would be possible to greatly increase the performance of the processor by pipelining 
its different paths and modules. Just by removing the execution sequence, the troughput would be 
multiplied by 7. Concerning pipelining, if it is done at all levels (especially for complex operations 
of the alu: multiplication and division), it would increase the clock frequency of the CPU. With 
this done correctly, it would certainly be possible to go up to 100 or 200 MHz, compared to the current 50MHz. Then, it 
is possible to add more IOs. At least managing all the ones on the card and adding audio management 
on the HDMI would be a nice addition. Finally, modifications to the software on the ARM side and 
memory access on the FPGA side could allow debugging with breakpoints in the code etc., which could 
be really interesting for the labs. 

Although the system is already functional and easy to use, there are still many possibilities for 
modifications. The only limit is the capacity of the FPGA, which could be changed anyway if needed.
