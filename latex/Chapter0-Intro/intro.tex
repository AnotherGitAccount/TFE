\chapter*{Introduction}

Although some liberties have been taken on the end of the work, the main objective of this thesis
is to provide a laboratory tool for the computation structures course. Indeed, the professors of 
the course needed a replica of the beta machine, an Harvard architecture based CPU that is learned
throughout the course. This machine had to be physically implemented so that the students could 
really take it in hand rather than using simulations. In order to design it in hardware, it was 
decided to work on an FPGA, a Cyclone V from Altera to be precise. In fact, everything was done on a 
development board, the DE10-Nano from Terasic. This allows to focus almost only on the logic and 
structural implementation rather than on all the electrical details intrinsic to such 
implementations.

This document therefore starts with a first chapter containing an enumeration of the different 
functionalities of this board as well as a description of the generalities concerning FPGAs. A quick 
comparison with other similar technologies is also made. Then, an overview of the two parts of the 
FPGA, starting with the programmable logic followed by the ARM processor part is done. The existing 
interconnection between these two parts is also described afterward. The last part of this chapter 
is a comparison between the different memories available on the board and on the FPGA.

After introducing the hardware, an introduction to the FPGA development flow is 
given. In this chapter, a brief discussion on the different specific tools used throughout this work 
is made. The specificities of the programming language used, Verilog, are also presented. The
chapter finally closes with a discussion on the compilation flow.

Now that the groundwork has been laid, the heart of the matter can be addressed. This is why the 
third chapter contains a description of the beta machine. In this description, the reader is 
made aware of the dimensioning of the machine: the word size, the endianness of the machine, its 
instruction set, etc. Following this, the implementation of the CPU on FPGA is presented in detail. 
An informal verification by simulation of the program counter and the alu is finally conducted.

Chapter 4 covers the design of the Memory Access Unit (MAU) which gives access to the memories of 
the beta machine from the ARM processor. The description of the Control Unit (CTRLU) allowing the 
startup and shutdown of the machine from the ARM processor is conducted here. The communication 
protocol common to both units is first presented before detailing the implementations. 

At this stage, the specifications are fulfilled for the hardware part of this master thesis. That
said, it would have been a shame to deliver this machine without any interface other than what is 
possible from the ARM processor. This is why the following chapter details the implementation of a 
graphics accelerator which will be abusively called GPU. The HDMI controller and the protocols 
allowing its use are introduced beforehand. The Clock unit (CLKU) is also presented just after a 
brief introduction to PLLs.

In order to provide easy access to the machine, an operating system must be installed on the ARM 
part of the FPGA. Chapter 6 discusses the choices made at this level and what has been tried 
for this project. A brief discussion of the booting process and the creation of images are included.

Then, a chapter is dedicated to the different tools developed to facilitate the access and the 
programming of the machine by the users.

Finally, the last chapter of this document lists different demonstrations programmed in assembler
that serve as proof of concept. The assembler libraries written for this work are also briefly 
presented.