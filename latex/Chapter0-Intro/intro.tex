\chapter*{Acknowledgements}

I am grateful to all those who helped and supported me during this work. In particular professors 
Fontaine Pascal and Mathy Laurent who were very regular in their follow-up by participating with me 
in regular meetings. In addition to that, they were available to answer all my questions during the 
whole duration of this work, which was very reassuring and helpful. I would like to thank Professor 
Fontaine in particular for his very careful review of this work and for the many pertinent 
suggestions that resulted from it. But also for putting me in touch with people who could guide me 
on certain choices for this work.

These two people are Nadel Alexander who is a research scientist at Intel Israel and Gyuszi 
Suto a principal engineer at Intel Oregon. I thank them greatly for the time they devoted to me through 
several emails and a video conference meeting during which I could present my work and discuss with 
them. They specifically helped me in the choices of how I could 
improve the confidence one could have in a Central Processing Unit (CPU) such as the one developed in this work.

I would also like to thank Mormont Romain, a PhD student at the University of Liège in machine learning
who provided an assembler software that outputs beta machine code. He also took 
the time to explain me how it works and how to use it by mail. One last person directly involved in 
this work is Schnackers Gauderic, a physics engineer who took the time to try out the system 
designed in this work by programming a 2D video game on it. I thank him for this and for his friendship.

Finally, I could not end my thanks without thinking of all my family and friends who have always been 
very supportive and who knew how to entertain me when I wanted to work too much.

\chapter*{Introduction}

Although some liberties have been taken at the end of the work, the main objective of this master thesis
is to provide a laboratory tool for the computation structures course INFO0012-2 at the University of Liège. Indeed, the professors of 
the course needed a replica of the beta machine, an Harvard architecture based CPU that is learned
throughout the course. This machine had to be physically implemented so that the students could 
really take it in hand rather than using simulations. In order to design it in hardware, it was 
decided to work on an Field Programmable Gate Array (FPGA), a Cyclone V from Altera to be precise. In fact, everything was done on a 
development board, the DE10-Nano from Terasic. This allows to focus almost exclusively on the logic and 
structural implementation rather than on all the electrical details intrinsic to such 
implementations.

This document therefore starts with a first chapter containing an enumeration of the different 
functionalities of this board as well as a description of the generalities about FPGAs. A quick 
comparison with other similar technologies is also made. Then, an overview of the two parts of the 
FPGA, starting with the programmable logic followed by the ARM processor part is done. The existing 
interconnection between these two parts is also described afterward. The last part of this chapter 
is a comparison between the different memories available on the board and on the FPGA.

After introducing the hardware, an introduction to the FPGA development flow is 
given. In this chapter, a brief discussion on the different specific tools used throughout this work 
is made. The specificities of the programming language used, i.e., Verilog, are also presented. The
chapter finally closes with a discussion on the compilation flow.

Now that the groundwork has been laid, the core of the matter can be addressed. This is why the 
third chapter contains a description of the beta machine. In this description, the reader is 
made aware of the dimensioning of the machine: the word size, the endianness of the machine, its 
instruction set, etc. Following this, the implementation of the CPU on FPGA is presented in detail. 
The IO unit which provides access to several IOs of the DE10-nano board to the CPU is also
described in this chapter.

At this stage, the specifications are fulfilled for the hardware part of this master thesis. That
said, it would have been a shame to deliver this machine without any interface other than what is 
possible using the IO unit. This is why the following chapter details the implementation of a 
graphics accelerator which will be abusively called GPU (Graphics Processing Unit). The HDMI controller and the protocols 
allowing its use are introduced beforehand. The clock unit (CLKU) is also presented.

Chapter 5 covers the design of the Memory Access Unit (MAU) which gives access to the memories of 
the beta machine from the ARM processor. The description of the Control Unit (CTRLU) allowing the 
startup and shutdown of the machine from the ARM processor is given here. The communication 
protocol common to both units is first presented before detailing the implementations. 

In order to provide easy access to the machine, an operating system must be installed on the ARM 
part of the FPGA. Chapter 6 discusses the choices made at this level.

Finally, a chapter is dedicated to the different tools developed to facilitate the access and the 
programming of the machine by the users. This last chapter also lists different demonstrations 
programmed in assembly language that serve as proof of concept. The assembly libraries written for 
this work are also briefly presented.

% PF: when you first use an acronym, define it
% QP: Ok
% PF: do not forget acknowledgements
% QP: Job done!
